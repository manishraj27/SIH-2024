\documentclass{article}
\usepackage{graphicx} % Required for inserting images

\title{SIH-2024}
\author{Manish Raj}
\date{August 2024}
\usepackage{amsmath}

\begin{document}

\maketitle

\section*{1. Carbon Emissions Estimation}
For each activity (e.g., excavation, transportation):
\[
\text{Total Emissions (CO}_2\text{)} = \sum (\text{Activity Data} \times \text{Emission Factor})
\]
- \textbf{Activity Data}: The measurable quantity of each activity (e.g., fuel consumed, distance traveled, hours of operation). \\
- \textbf{Emission Factor}: The amount of CO$_2$ emitted per unit of activity (e.g., kg CO$_2$ per liter of diesel).

Example:
\[
\text{CO}_2\text{ Emissions from diesel use} = \text{Liters of diesel used} \times \text{Emission factor (kg CO}_2/\text{liter})
\]

\section*{2. Per Capita Emissions Estimation}
\[
\text{Per Capita Emissions (CO}_2\text{)} = \frac{\text{Total Emissions (CO}_2\text{)}}{\text{Number of People}}
\]
- \textbf{Total Emissions (CO$_2$)}: The sum of all emissions from various activities. \\
- \textbf{Number of People}: Total number of employees or residents associated with the mining operations.

\section*{3. Afforestation Offsets Calculation}
\[
\text{Carbon Offset (CO}_2\text{)} = \text{Area of Land} \times \text{Carbon Sequestration Rate (tons CO}_2/\text{hectare/year})
\]
- \textbf{Area of Land}: The amount of land dedicated to tree plantation. \\
- \textbf{Carbon Sequestration Rate}: The rate at which trees absorb CO$_2$, depending on tree species, soil type, and climate.

To calculate the area required for a target offset:
\[
\text{Required Area (hectares)} = \frac{\text{Total CO}_2\text{ Emissions to Offset (tons/year)}}{\text{Carbon Sequestration Rate (tons CO}_2/\text{hectare/year})}
\]

\section*{4. Carbon Credits Estimation}
\[
\text{Carbon Credits (tons CO}_2\text{)} = \text{Emissions Reduced or Offset (tons CO}_2\text{)}
\]
- \textbf{Market Value of Carbon Credits}:
\[
\text{Market Value} = \text{Carbon Credits (tons CO}_2\text{)} \times \text{Market Price (currency/ton CO}_2\text{)}
\]
- \textbf{Emission Reduction Strategies}: Can include adoption of cleaner technologies or afforestation, which reduce or offset emissions.

\section*{5. Energy Consumption Reduction}
For alternative energy sources:
\[
\text{Energy Consumption Reduction (kWh)} = \sum (\text{Traditional Energy Use} - \text{Renewable Energy Use})
\]
- \textbf{Traditional Energy Use}: The energy used by conventional sources (e.g., coal, diesel). \\
- \textbf{Renewable Energy Use}: The energy provided by renewable sources (e.g., solar, wind).

\section*{6. Gap Analysis between Emissions and Carbon Sinks}
\[
\text{Gap (tons CO}_2\text{)} = \text{Total Emissions (tons CO}_2\text{)} - \text{Total Carbon Sinks (tons CO}_2\text{)}
\]
- \textbf{Total Emissions}: The total emissions calculated from various activities. \\
- \textbf{Total Carbon Sinks}: The total amount of CO$_2$ absorbed by carbon sinks (e.g., forests, soil).

\section*{7. Scalability and Emission Estimation for Different Mine Sizes}
For scalability, the same emission estimation formula can be applied but adjusted based on the size of the mine:
\[
\text{Adjusted Emissions (tons CO}_2\text{)} = \text{Total Emissions (tons CO}_2\text{)} \times \text{Scaling Factor}
\]
- \textbf{Scaling Factor}: A ratio based on the size and type of the mine (e.g., larger mines may have a higher scaling factor).


\end{document}
